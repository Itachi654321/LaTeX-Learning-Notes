%导言区
\documentclass{book}%文章类型

\usepackage[heading=true]{ctex}%中文宏包
\usepackage{CJKfntef}%中文下划线宏包
\usepackage{xltxtra}%XelaTeX标志符号
\usepackage{amsmath}%数学公式
\usepackage{graphicx}%图像
\usepackage{titlesec}%标题格式
\usepackage{fancyhdr}%添加页眉页脚
\usepackage{fancyvrb}%抄录环境
\usepackage{texnames}%标志符号
\graphicspath{figures/}%图像路径
\CTEXoptions[today=old]%日期格式
\usepackage{ulem}
\usepackage{amssymb}%数学宏包
\usepackage{geometry}%页面设置
\usepackage{makecell}%表格内换行
\usepackage{alltt}%抄录环境
\geometry{centering}%版心居中
\geometry{top=2cm}%上边缘宽度
\geometry{bottom=2cm}%下边缘距离
\geometry{left=2cm}%左边缘距离
\geometry{right=2cm}%右边缘距离

%正文区
\begin{document}

	\setlength{\parindent}{0pt}%取消首行缩进
	
	\begin{titlepage}%自定义标题格式
	\vspace*{\fill}
	\begin{center}
		\normalfont
		{\huge\bfseries \LaTeX }
		
		\bigskip
		{\Large\itshape Itachi酱\\木叶村暗部}
		
		\medskip
		\today
	\end{center}
    \vspace{\stretch{3}}
	\end{titlepage}

    \thispagestyle{fancy}

	\newpage%生成目录页
	\let\cleardoublepage\clearpage%清除空白页	

	\tableofcontents%输出目录	
	
	\chapter{写在学前}
	
	\section{我的\LaTeX 资源}
	
	\subsection{GoodNotes}
	简单高效\LaTeX\\
	一份(不太)简短的\LaTeX 介绍\\
	A Practical Guide to \LaTeX\enspace Tips and Tricks\\
	China\TeX 培训期刊\\
	\CTeX 宏集手册\\
	\LaTeX\enspace Beginner's Guide\\
	\LaTeX\enspace Cookbook\\
	\LaTeX\enspace Notes\\
	\LaTeX 科技论文写作简明教程\\
	\LaTeX 快速入门与提高\\
	\LaTeX 排版学习笔记\\
	\LaTeX 入门\\
	\LaTeX 完全学习手册\\
	\LaTeX 文类和宏包学习手册\\
	Math mode\\
	Mathematics into Type\\
	More Math Into \LaTeX\\
	Practical \LaTeX\\
	\TeX\enspace for the Impatient\\
	\TeX 主题演讲\\
	The Art of \LaTeX\\
	The BEAMER class\\
	The \LaTeX\enspace Companion\\
	The \LaTeX\enspace Mathematics Companion\\
	
	\subsection{Notability}
	表格手册\\
	绘图教程\\
	介绍\\
	索引\\
	文献\\
	中文使用手册\\
	beamer\\
	\CTeX\\
	graphicx\\
	mathematics\\
	physics\\
	
	\subsection{百度网盘}
	视频\\
	书籍\\
	资料\\
	
	\subsection{知乎}
	LaTeX(未读)\\
	LaTeX(已读)\\
	
	\subsection{网易公开课}
	LaTeX(未看)\\
	LaTeX(已看)\\
	
	
	\subsection{bilibili}
	LaTeX(未看)\\
	LaTeX(已看)\\
	
	\subsection{移动硬盘}
	\LaTeX 教程\\

    \subsection{XMind}
    
    \subsection{网站}
    CTAN\\
    水木社区TEX版\\
    LaTeX编辑部\\
    Cambridge Notes\\
    Creat LaTeX tables online\\
    Slager\\
    TEX.SE\\
    GitHub\\
    Jacky's site\\
    Mathcha\\
    
	\section{我的学习指南}
	学习笔记上传知乎。\\
	学习\LaTeX ,可用于学习笔记、作业、论文写作、国赛、美赛。\\
	在记笔记的过程中不断完善思维导图。\\
	
	\section{问题、想法汇总}
	如何用Python编辑脚本编写LaTeX文档中重复的内容。\\
	alltt环境不起作用。\\
	verb抄录环境中如何换行。\\
	如何用LaTeX设计版面和宏。\\
	
	\section{我的学习日常}
	2022.3.3开始学习\LaTeX。\\
	\begin{table}[h]
		\begin{tabular}{|c|c|}
			\hline
			2022.3.3&学习LaTeX入门\\
			\hline
			2022.3.4&学习LaTeX入门、制作LaTeX思维导图、写知乎文章\\
			\hline
		\end{tabular}
	\end{table}

	\chapter{构建文档的基本过程}
	基本介绍:LaTeX是国际上科技领域展业排版的实际标准。\\
	
	\section{导言区}
	(1)声明文章类型:\\
	\textbackslash documentclass \{文章类型:book,reporter,article,letter,ctexart,ctexbook,ctexreport等\};\\
	文档类型ctexar前的UTF-8用于声明中文所用的编码格式。\\
	(2)引入宏包:\\
	\textbackslash usepackage\{宏包:ctex,amsmath,graphicx等\};\\
	(3)设置标题、作者、日期等:\\
	\textbackslash title/author/date\{标题、作者、日期\}。\\
	标题的产生分为两步,首先在导言区声明,然后在正文区输出。\\
	当文档类型选择cetxart,ctexbook,ctexreport时,不需要使用ctex宏包。
	
	\section{正文区}
	(1)\textbackslash begin\{document\}(标识正文范围);\\
	(2)输出标题:\\
	\textbackslash maketitle;\\
	(3)输出目录:\\
	\textbackslash tableofcontents;\\
    (4)分章节写正文:\\
    \textbackslash chapter/section/subsection/subsubsection\{章节标题\};\\
    (5)\textbackslash end\{document\}。
          
    \section{增加目录项目}
    \textbackslash 使用tocbibind宏包:\\
    \textbackslash usepackage[nottoc]\{tocbibind\}\\
    默认在目录中加入目录项本身、参考文献、索引等。\\
    
    \chapter{字体属性}
    
    \section{字体族}
    \textbackslash textrm(装饰)/sf(无装饰)/tt(等宽)\{文本\};\\
    \{\textbackslash rm/sf/ttfamily 文本\}。
    
    \section{字体系列}
    \textbackslash textmd/bf\{文本\};\\
    \{\textbackslash md/bfseries 文本\}。
    
    \section{字体形状}
    \textbackslash textup/it/sl/sc\{文本\};\\
    \{\textbackslash up(直立)/it(斜体)/sl(伪斜体)/sc(小型大写)shape 文本\};\\
    \{\textbackslash songti/heiti/kaishu/fangsong 文本\}。\\
    
    \section{字体大小}
    \{tiny/scriptsize/footnotesize/small/normalsize/large/Large/LARGE/huge/Huge 文本\};\\
    \textbackslash zihao\{字体大小\} 文本,中文字号可设置为-6,-5,$\ldots$,-0,0,$\ldots$,8。\\
    
    \section{自定义引用字体}
    \textbackslash newenvironment\{myquote\}\\
    \{\textbackslash begin\{quote\}字体族\textbackslash zihao\{字号\}\}\\
    \{\textbackslash end\{quote\}\}\\
    自定义命令环境中第一个参数是环境的名字,后两个参数是环境开始和结束的代码。
    
    \section{强调}
    \textbackslash emph命令表示强调。
    
    \chapter{页面设置}
    
    \section{长度单位}
    pt,pc(1pt=12pc),in(1in+72.27pt),bp,cm,mm,dd,cc,sp(最小长度单位),em,ex。
    
    \section{缩进}
    禁用缩进:在导言区使用\textbackslash noindent命令可取消首行缩进。但此时\textbackslash par后面的内容仍然默认缩进。如果想要禁止\textbackslash par后面自动缩进,可在导言区使用\textbackslash par\textbackslash noindent命令。\\
    产生缩进:\textbackslash indient\\
    缩进长度:\textbackslash parindient\\
    悬挂缩进:\textbackslash hangafter、\textbackslash hangindient\\
    首字下沉:使用lettrine宏包。\\
    悬挂缩进:悬挂缩进只需要在需要设置缩进的段落上下使用\textbackslash hangafter=需要缩进的长度即可。\\
    自定义缩进:如果想要自定义缩进长度,可在导言区使用\textbackslash setlength \textbackslash parindident\{需要所经的长度\}命令。\\
    
    \section{分段换行}
    \textbackslash \textbackslash 用于换行;\\
    \textbackslash par用于分段;\\
    空行用于产生新的段落。
           
    \section{段落对齐}
    \textbackslash raggedright\\
    \textbackslash raggedleft\\
    对齐环境:center,flushleft,flushright
       
    \section{清除空白页}
    在某命令之后出现空白页,则在该命令之前添加该命令:\textbackslash let\textbackslash cleardoublepage\textbackslash clearpage。
    
    \section{页面尺寸}
    使用geometry宏包:\\
    \textbackslash usepackage\{geometry\}\\
    \textbackslash geometry\{纸面大小(如a6paper),版心对齐方式(如centering),长宽占页面的比例(如scale=0.8)\}\\
          
    \section{脚注}
    \textbackslash footnote命令。
    
    \section{空行}
    空行前需要先换行。\\
    \~ \thinspace \textbackslash \textbackslash\\
    控制间距的空行:\textbackslash \textbackslash []
    
    \section{行距}
    \textbackslash linespread\{实际行距为基本行距的倍数\}  
        
    \section{盒子}
    盒子是TEX中的基本处理单位,一个字符、一行文字、一个页面、一张表格在TEX中都是一个盒子。\\
    水平盒子:\\
    \textbackslash mbox\\
    \textbackslash makebox[宽度][位置,包括l,r,c,s等]{内容}\\
    \textbackslash fbox:带边框的盒子\\
    \textbackslash framebox[宽度][位置,包括l,r,c,s等]{内容}\\
    调整边框线的粗细:\verb|\setlength{\fboxrule}{...}|\\  
    内边距:\verb|\setlength{\fboxsep}{...}|\\
    垂直盒子:\\
    \verb|\parbox[<位置>][<高度>][<内容位置>]{<宽度>}{<内容>}|\\
    \verb|\begin{minipage}[<位置>c(居中)、t(顶部)、b(底部)、s(分散)][<高度>][<内容位置>]{<宽度>}抄录内容\end{minipage}|\\
        
    \chapter{文本环境}
    
    \section{引用环境}
    qoute,qoutation,verse环境。\\
    使用quote环境,其中内容单独分行,增加缩进和上下间距排印,不改变字体。\\
    基本的交叉引用命令是\textbackslash ref,以标签为参数,得到被引用的编号。\\
    公式环境也可进行交叉引用,\textbackslash begin\{equation\}\textbackslash label\{标签名称\}\\
    amsmath宏包中包括公式的引用命令\textbackslash eqref。
    
    \section{摘要环境}
    abstract环境\\
    在\textbackslash maketitle之后用abstract环境生成。\\
    
    \section{列表环境}
    eumerate、itemize、description环境
    
    \section{定理环境}
    \verb*|\newtheorem{环境名称}{定理名称}|\\
    定理环境格式:\textbackslash theoremstyle\{plain、break、marginbreak、changebreak、change、margin\}
    
    \section{抄录环境}
    \textbackslash verb命令、verbetim环境\\
    alltt宏包可引入抄录环境。\\

    \chapter{图表}
    
    \section{图像}
    导言区引入graphicx宏包,并用graphicspath\{图像文件夹名称/\}设置好图像文件路径后,在figure浮动环境中,使用\textbackslash includegraphics[\textbackslash width=\textbackslash linewidth,\textbackslash angle=旋转角度,\textbackslash scale=缩放大小]\{图像文件名称\}插入图像。\\
    插入的图形本质上就是一个矩形盒子。\\
    通常把图形放在一个可以变换相对位置的环境中,这种环境称为浮动体。\\
    
    \section{表格}
    在table浮动环境中,使用tabular\{格式\}环境生成表格,竖线由|产生,横线由\textbackslash hline产生。行与行用\textbackslash \textbackslash 分割,列与列用\&分割。\\
    tabular环境之后有一个必选参数,里面声明了表格中列的格式。
      
    \section{标题}
    \textbackslash caption\{图表名称\}设置图标标题。
    
    \section{标签引用}
    \textbackslash lable\{标签名称\}打上标签;\\
    \textbackslash ref\{标签名称\}引用。
    
    \section{浮动环境位置}
    浮动环境可选参数:h当前位置,t页顶,b页底,p独占一页。
    
    \section{不浮动的图表环境}
    [H]是由float宏包提供的特殊功能,表示就放在这里,不浮动,可置于\textbackslash begin\{浮动环境\}之后。、
    
    \section{改变图表标题格式}
    使用caption宏包:\\
    \textbackslash usepackage[标题对齐方式(如format=hang),字号(如font=small),标题文本类型(如textfont=it)]\{caption\}
    
    \chapter{符号}
    
    \section{空白符号}
    \textbackslash quad(1em)、\textbackslash qquad(2em)、\textbackslash thinspace(1/6em)、\textbackslash ensapce(0.5em)。\\
    不要使用中文的全角字符。\\
    \LaTeX 会自动忽略每行开始的空格,通常汉字后面的空格会被忽略,英文和其他符号后面的空格会被保留。\\
    注意用空格把命令和后面的中文隔开。\\
    段首不被忽略的空白:使用\textbackslash hspace*\{距离\}命令。\\
    通常汉字后面的空格会被忽略,其他符号后面的空格则被保留。\\
    
    \section{控制符}
    \textbackslash \#、\textbackslash \$、\textbackslash \&、\textbackslash \%、\textbackslash \{、\textbackslash \}、\textbackslash \{\textbackslash\}、\textbackslash textbakslash。  
    
    \section{希腊字母}
    \textbackslash alpha、\textbackslash beta、\textbackslash epsilon、\textbackslash pi、\textbackslash omega
    、\textbackslash Gamma、\textbackslash Delta、\textbackslash Omega、\textbackslash Theta、\textbackslash Pi等。
    
    \section{标志符号}
    \textbackslash Tex、\textbackslash LaTeX、\textbackslash LaTeXe。
    
    \section{连字符}
    由一个、两个、三个连续的减号生成短、中、长三种连字符。
    
    \section{引号}
    数字键1左边的撇号表示左单引号,单引号字符表示右单引号,连续两个撇号表示左双引号,连续两个单引号字符表示右双引号。
    
    \section{省略号}
    \textbackslash cdots居中的省略号\\
    \ldots/\dots:西文省略号由\textbackslash ldots、\textbackslash dots产生。
    
    \section{排版符号}
    $\S$:\$\textbackslash S\$。
    
    \section{下划线}
    \textbackslash underline命令\\
    更多的下划线使用ulem宏包;\\
    \uline{uline}\quad\uuline{uuline}\quad\uwave{uwave}\quad\sout{sout}\quad\xout{xout}\quad\dashuline{dashuline}\quad\dotuline{dotuline}\\
    中文下划线需要使用CJKfntef宏包:\\
    
    \chapter{数学公式}
    数学公式分为行内公式、行间公式和列表公式。\\
    
    \section{常用数学宏包}
    \verb|\usepackage{amsmath, amssymb}  % standard packages for math writing,\usepackage{listings}   % include the contents of code files ,\usepackage{mathpazo} % a better font than the default,\usepackage{mathtools},\usepackage{enumitem},\usepackage{amsthm}|\\
    
    \section{行内公式}
    (1)\$数学模式\$;\\
    (2)\textbackslash(数学模式\textbackslash);\\
    (3)math环境。
    
    \section{行间公式}
    (1)\$\$数学模式\$\$;\\
    (2)\textbackslash []\textbackslash;\\
    (3)displaymath环境。
    
    \section{编号行间公式}
    使用equation环境。
    
    \section{非编号行间公式}
    使用equation*环境。
    
    \section{对齐公式}
    \verb|当你的公式不止一行,为了让可读性增强减少阅读障碍,我们需要用到对齐和换行。在 LaTeX 中,& 是对齐,\\ 是换行。|\\
    需要在align环境中使用。\\
        
    \section{上下标}
    行内公式中上下标默认在右下角,行间公式中上下标默认在正下方。\\
    上下标需要在数学模式中使用。\\
    \textbackslash \_\{下标\}命令;\\
    \textbackslash \^\quad\{上标\}命令。\\
    行内模式显示为行间模式使用\textbackslash diaplaystyle命令。\\
    上下标的叠加:使用\textbackslash atop命令,也可以使用\textbackslash substack\{下标\}叠加下标,各层之间使用\textbackslash \textbackslash 分割\\
    limits要求前面的必须为数学符号,不然会报错,解决方法是在非数学符号前使用\textbackslash mathop命令。
    
    \section{根式}
    \textbackslash sqrt\{\}。
    
    \section{分式}
    \textbackslash frac\{\}\{\}前为分子,后为分母。
    
    \section{算符}
    \verb|加减乘除 +、−、∗、/ 可直接输入,乘号 × \times,除号 ÷ \div,点乘 · \cdot,加减号 ± \pm / ∓ \mp|
    \textbackslash log/sin/cos/tan/cot/arcsin/arccos/ln。
    
    \section{极限}
    极限符号:\$\textbackslash lim\_\{\textbackslash to\}\$\\
    无限:\textbackslash infty\\
    将极限符号置于正下方:\textbackslash displaystyle(行内转换为行间格式);\$\textbackslash lim\textbackslash limits\_\{\}\$。
    
    \section{角度}
    $^\circ$:\$ \^ \enspace \textbackslash circ\$    
    
    \section{矩阵}
    行与行之间用双斜杠\textbackslash \textbackslash 分割,列与列之间用\& 分割。\\
    不带括号的矩阵使用matrix环境,带圆括号的矩阵使用pmatrix环境,方括号矩阵使用bmatrix环境,大括号矩阵使用Bmatrix环境,竖线矩阵使用vmatrix环境,双竖线矩阵使用Vmatrix环境。\\
    还可以使用array环境,与tabular环境类似,必选参数表示列的对齐方式。
    
    \section{关系符}
    \verb|=,>,<,直接输入,不等号≠ \ne,大于等于号 ≥ \ge,小于等于号 ≤ \le,约等号 ≈ \approx,等价 ≡ \equiv|\\
    小于等于:\textbackslash leq\\
    大于等于:\textbackslash geq\\
    
    \section{箭头}
    \verb|常用的箭头包括→ (\rightarrow 或 \to)、← (\leftarrow 或 \gets)|\\
    
    \section{括号和定界符}
    \verb|LaTeX 提供了多种括号和定界符表示公式块的边界,如小括号 ()、中括号 []、大括号 {} (\{\})、尖括号 ⟨⟩ (\langle \rangle)、上括号(\overbrace)、下括号(\underbrace)等。|\\
    调整括号的大小:\verb|\bigl( \Bigl( \biggl( \Biggl( \quad \bigr\} \Bigr\} \biggr\} \Biggr\} \quad|\\
    
    \section{定理环境}
    在使用前需要在导言区定义:\\
    \textbackslash newtheorem\{thm\}\{定理\}\\
    定理环境有一个可选参数,就是定理的名字。
    
    \section{自定义}
    集合符号编写麻烦,可在导言区进行简化定义:\\
    \verb|\def\ZZ{{\mathbb Z}},\def\RR{{\mathbb R}},\def\CC{{\mathbb C}},\def\QQ{{\mathbb Q}},\def\EE{{\mathbb E}}|\\
    简化尖括号符号:\\
    \verb|\newcommand{\la}{\langle},\newcommand{\ra}{\rangle}|\\

    \section{证明环境}
    在证明环境之后添加小正方形:\\
    \verb|\newenvironment{proof}{\emph{Proof.}}{\hfill$\square$}|\\
    
    \section{引用}
    数学公式的引用需要使用amsmath宏包中的eqref命令。\\
    
    \chapter{文献列表}
    常用的文献管理软件是Zotera,为实现更好的功能需要安装Better BibTeX插件。\\ 
    一次管理,一次使用:使用thebibliography环境和\textbackslash bibitem命令。\\
    一次管理,多次使用:用bib文献库,\textbackslash cite命令引用。
    用\textbackslash bibliographystyle命令声明参考文献的格式,用\textbackslash bibliography 打印参考文献列表。\\
    每则文献包括类型、引用标签、标题、作者、出版社、出版年等。\\
    使用文献管理工具JabRef制作参考文献数据库文件。\\
    \BibTeX 处理文献文件运行四次,第一次准备辅助文件,第二次生成\LaTeX 代码,后两次读入文献列表代码并生成正确的信息。\\
    \LaTeX 只选择被\textbackslash cite 引用的文献,并在引用位置显示在文献列表中的编号,如果要显示并不直接引用的文献,需要在\textbackslash bibliography之前使用\textbackslash nocite 命令。\\
    BibTeX使用的参考文献数据库其实是一个后缀名为.bib的文件。\\
    
    \chapter{幻灯片}
    
    \section{简介}
    Beamer是一个用于制作演示文稿的\LaTeX 文档类。\\
    文档类型选择beamer,每张幻灯片都是放在一个frame环境中的。\\
    在载入ctex宏包的时候加上noindient选项可以取消段落缩进。\\
    每张幻灯片可以添加标题和副标题在必选参数中。\\
    使用\textbackslash frametitle\{标题\}、\textbackslash framesubtitle\{副标题\}设置标题。\\
    内容位置可以通过可选参数[t,c,b]设置。\\
    
    \section{目录小节}
    使用\textbackslash titlepage生成标题页。\\
    可选参数plain表示不显示外部元素。\\
    可使用part,section,subsection等结构命令。\\
    \textbackslash tableofcontents产生目录页面,可选参数hideallsubsections表示不显示小节标题。
    
    \section{环境}
    
    \subsection{区块环境}
    block
    
    \subsection{提醒环境}
    alertblock
    
    \subsection{例子环境}
    exampleblock
    
    \subsection{证明环境}
    proof\\
    证明二字改变为中文,\textbackslash renewcommand\{proofname\}\{证明\}
    
    \subsection{分栏环境}
    columns环境和column命令,分栏对齐在columns环境之后使用onlytextwidth选项。    
    
    \section{抄录环境}
    使用verb抄录环境时,需要在frame环境之后加上fragile可选参数
    
    \section{创建覆盖}
    \textbackslash pause/onslide等\\
    半透明显示:\textbackslash setbeamercovered\{transparent/invisible/dynamic/highly dynamic\}
    
    \section{主题}
    
    \section{选用}
    使用\textbackslash suetheme\{主题名\}选择整体主题;使用\textbackslash usebeamerouter/inner/color/fonttheme选择细分主题。
    
    \section{定制}
    
    三种标准文类的章节命令与层次深度\\

\end{document}